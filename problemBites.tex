\documentclass[reqno]{amsart}
\usepackage{amsmath, amssymb, fullpage}
\title{Some Cool Problems And Their Solutions}
\author{Nivas}

\newcommand{\rl}{\mathbb{R}}
\newcommand{\rls}[1]{\mathbb{R}^{#1}}
\DeclareMathOperator{\var}{Var}
\newcommand{\variance}[1]{\var \left[ #1 \right]}
\newcommand{\E}[1]{\mathbb{E}\left[ #1 \right]}
\newcommand{\Prb}[1]{\mathbb{P} \left[ #1 \right]}

\renewcommand{\a}{\alpha}
\newcommand{\e}{\epsilon}
\renewcommand{\t}{\tau}

\renewcommand{\a}{\alpha}
\renewcommand{\b}{\beta}
\newcommand{\g}{\gamma}
\renewcommand{\d}{\delta}
\renewcommand{\e}{\epsilon}
\newcommand{\f}{\phi}
\renewcommand{\r}{\rho}
\renewcommand{\t}{\theta}


\begin{document}
\maketitle
\newpage
\section{Probability}
\subsection{Handling Infinite Chains}
What is the mean number of times we need to roll a fair die to land a 6?

\subsubsection{Summing an AGP}
This is a simple problem with a few different ways to solve it. The first way is straightforward, involving a sort of insight that is a bit removed from the problem itself. Suppose $m$ is the average number of rolls to get a 6. Clearly, the chance of getting a 6 on the first roll is $1/6$. The chance of getting it on the second roll is $\displaystyle \frac{5}{6} \frac{1}{6}$, and on roll $i$, is $\displaystyle \frac{5}{6}^{i-1} \frac{1}{6}$. Averaging, 
\[ m = \sum_{i=0} ^{\infty} (i+1) \frac{5}{6}^{i} \frac{1}{6}. \]
We note that $m$ is the scalar product between an arithmetic and geometric progression, or succinctly: AGP. The problem reduces to summing this infinite AGP. For this, we employ the same trick we use for summing a GP, twice over: the first time to obtain a GP and the second time to obtain the sum of this GP. Let $r = 5/6$ and $a = 1/6$.
\[  m' := m r = \sum (i+1) r^{i+1} a, \quad \Rightarrow m' - m = a + \sum_{i=1} r^i a = a \left( 1 + \sum_{i=1} r^i \right) = a \sum_{i = 0} r^i . \]
Since the final sum is just a GP, we employ the standard trick to get 
\[ m - m' = (1 - r)m = \frac{a}{1 - r},  \]
whence
\[ m = \frac{a}{(1 - r)^2} = 36/6 = 6. \]

\subsubsection{Pinching The Beginning}
There is another, more beautiful way to solve the problem. Note that, if we don't get a 6 on the first throw, we are in a position exactly as we were in the beginning, before the first throw, and it'll take $m$ throws on average to land a 6. If we do land a 6 on the first throw, then we're done. Weighting these events with the corresponding probabilities, 
\[ m = \frac{1}{6} + (m+1)\frac{5}{6} = \frac{5}{6}m + 1, \]
implying that
\[ \frac{m-1}{m} = \frac{5}{6} \Rightarrow m = 6.  \]
\end{document}